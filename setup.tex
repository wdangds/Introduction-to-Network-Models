\usepackage[mmddyyyy]{datetime}	
\usepackage[margin=0.7in]{geometry}		% For setting margins
\usepackage{amsmath}				% For Math
\usepackage{fancyhdr}				% For fancy header/footer
\usepackage{graphicx}				% For including figure/image
\usepackage{cancel}					% To use the slash to cancel out stuff in work
\usepackage{subcaption}
\usepackage{makecell}
\usepackage{blindtext}
\usepackage{xcolor}
\usepackage{enumitem}
\usepackage{listings}
\usepackage{algorithm}
\usepackage{algpseudocode}
\usepackage{xcolor}
\usepackage{longfbox} 
\usepackage[colorlinks=true, linkcolor=blue, citecolor=teal, urlcolor=magenta]{hyperref}
\usepackage{booktabs}
\usepackage{amsmath,amssymb,amsfonts}
\usepackage{tabularx}
\usepackage{listings}
\usepackage{xcolor}
\usepackage{array}
\usepackage{multirow}
\usepackage{tikz}
\usetikzlibrary{positioning, fit, backgrounds}
\usepackage{amsthm}

\definecolor{codegreen}{rgb}{0,0.6,0}
\definecolor{codegray}{rgb}{0.5,0.5,0.5}
\definecolor{codepurple}{rgb}{0.58,0,0.82}
\definecolor{backcolour}{rgb}{0.95,0.95,0.92}


%%%%%%%%%%%%%%%%%%%%%%
% Set up fancy header/footer
\pagestyle{fancy}
\fancyhead[LO,L]{Introduction to Network Model}
\fancyhead[CO,C]{}
\fancyhead[RO,R]{\today}
% \fancyfoot[LO,L]{aivietnam.edu.vn}
% \fancyfoot[CO,C]{}
% \fancyfoot[RO,R]{www.facebook.com/aivietnam.edu.vn}
\renewcommand{\headrulewidth}{0.4pt}
\renewcommand{\footrulewidth}{0.4pt}
%%%%%%%%%%%%%%%%%%%%%%
%%%%%%%%%%%%%%%%%%%%%%
\usepackage[most]{tcolorbox}  % For the tcolorbox environment
\usepackage{algorithm}
\usepackage{algpseudocode} % Modern replacement for algorithmic
\usepackage{colortbl}
\usepackage{multicol} % Allows multiple columns
\usepackage{lipsum}
\usepackage[backend=biber,style=ieee]{biblatex} % Use Biber and specify the citation style
\addbibresource{references.bib} % Link to your .bib file
%%%%%%%%%%%%%%%%%%%%%%
\newcommand{\annotate}[1]{\marginpar{\footnotesize\color{gray}#1}}
\NewDocumentCommand{\define}{m o}{\emph{#1}\annotate{\IfValueTF{#2}{#2}{#1}}}
%%%% Definition
% ---- Preamble setup ----
 % loads theorems, skins, breakable, etc.
\tcbuselibrary{theorems, skins, breakable}
\usepackage[nameinlink,noabbrev]{cleveref}

% Base look & feel for all theorem boxes
\tcbset{
  mytheo/.style={
    enhanced,
    breakable,
    coltitle=black,
    fonttitle=\bfseries,
    boxed title style={boxrule=0pt},
    title after break=\textit{(continued)}}
}

% ---- Environments ----
% Number within section; change to number within=chapter if you prefer
\newtcbtheorem[number within=section,crefname={definition}{definitions}]%
  {definition}{Definition}%
  {mytheo,colback=blue!3,colframe=blue!80!black!20,boxed title style={colback=blue!12}}%
  {def}

\newtcbtheorem[number within=section,crefname={properties}{properties}]%
  {properties}{Properties}%
  {mytheo,colback=purple!3,colframe=purple!60!black!20,boxed title style={colback=purple!12}}%
  {prop}

\newtcbtheorem[number within=section,crefname={example}{examples}]%
  {example}{Example}%
  {mytheo,colback=orange!3,colframe=orange!70!black!20,boxed title style={colback=orange!15}}%
  {ex}

  \newtcbtheorem[number within=section,crefname={tip}{tips}]%
  {tip}{Tip}%
  {mytheo,
   colback=green!3,
   colframe=green!60!black!20,
   boxed title style={colback=green!12},
   title={Tip: #2}}%
  {tip}


% listings settings
\lstdefinestyle{mystyle}{
    backgroundcolor=\color{backcolour},   
    commentstyle=\color{codegreen},
    keywordstyle=\color{magenta},
    numberstyle=\tiny\color{codegray},
    stringstyle=\color{codepurple},
    basicstyle=\ttfamily\footnotesize,
    breakatwhitespace=false,         
    breaklines=true,                 
    captionpos=b,                    
    keepspaces=true,                 
    numbers=left,                    
    numbersep=5pt,                  
    showspaces=false,                
    showstringspaces=false,
    showtabs=false,                  
    tabsize=2
}

\lstset{style=mystyle}